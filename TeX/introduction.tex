% !TeX root = ./document.tex
\chapter*{Úvod}
\addcontentsline{toc}{chapter}{Úvod}

Přednášející:
\begin{itemize}
    \item Petr Kaplický, KMA
    \item kaplicky@karlin.mff.cuni.cz
    \item www.karlin.mff.cuni.cz/$\sim$kaplicky
\end{itemize}

Literatura:
\begin{itemize}
    \item J. Kopáček: Matematická analýza (nejen) pro fyziky I (II) + příklady
    \item J. Souček: www.karlin.mff.cuni.cz/soucek
    \item V. Jarník: Diferenciální počet I
    \item V. Jarník: Integrální počet I
    \item W. Rudin: Principles of MA
    \item I. Černý, M. Rokyta: Differential and integral calculus of one real variable
\end{itemize}
\semester

\section{Diferenciální počet}
Mějme funkci $f(t)$ vyjadřující pozici bodu v čase. Základní úloha:

\begin{alignat}{2}
    \text{průměrná rychlost:}&\quad  &   &\frac{f(t) - f(t_0)}{t - t_0} \\
    \text{okamžitá rychlost:}&\quad  &   \lim_{t \to t_0}&\frac{f(t) - f(t_0)}{t - t_0} = f'(t_0)
\end{alignat}

\section{Integrální počet}
Plocha pod grafem. Interval $[a, b]$ rozdělme na $n$ částí délky $\Delta_n$ v bodech $a_n$.
Označme $a_0 = a$, $a_n = b$.

\begin{align}
    \text{přibližně:}\quad  &   f(a_0)\Delta_1 + f(a_1)\Delta_2 + \text{...} +
        f(a_{n-1})\Delta_n = \nonumber \\
    &=~S(\Delta) = \sum_{j=1}^{n}f(a_{j-1})\Delta_j \\
    \text{přesně:}\quad &   \lim_{\Delta \to 0} S(\Delta) = \int_{a}^{b}f(x)dx
\end{align}
