\chapter{Úvod, základní pojmy}
Výrok - má pravdivostní hodnotu 0 nebo 1. Mějme A, B výroky:

\begin{figure}[ht!]
	\begin{center}
		\begin{tabular}{|C|C|C|C|C|C|C|}
			\hline
			& & A\land B & & & (A\Rightarrow B) \land (A\Leftarrow B) & \\
			A & B & A\&B & A\lor B & A\Rightarrow B & A\Leftrightarrow B & \lnot A \\
			\hline
			0 & 0 & 0 & 0 & 1 & 1 & 1 \\
			0 & 1 & 0 & 1 & 1 & 0 & 1 \\
			1 & 0 & 0 & 1 & 0 & 0 & 0 \\
			1 & 1 & 1 & 1 & 1 & 1 & 0 \\
			\hline
		\end{tabular}
		\caption{Tabulka pravdivostních hodnot}
	\end{center}
\end{figure}

Důkaz implikace $A\Rightarrow B$:
\begin{enumerate}
	\item přímý: ukážeme, že když $A = 1$, pak $B = 1$
	\item nepřímý: plyne z $\lnot B \Rightarrow \lnot A$
	\item sporem: předpokládáme, že $A = 1 \land B = 0$ a odvodíme spor (např.: $1=2$)
\end{enumerate}

\begin{lemma}[Čtverec lichého čísla]
	(tvrzení) $\forall n\in \mathbb{N} : n^2~\text{liché} \Rightarrow n~\text{liché}$
\end{lemma}

\begin{proof}[Důkaz 1]
	Fixuj $n\in \mathbb{N}$. Prvočíselný rozklad:XXX
	\begin{gather}
		n = p_1^{\alpha_1} \cdot \ldots \cdot p_k^{\alpha_k} \\
		n^2 = p_1^{2\alpha_1} \cdot \ldots \cdot p_k^{2\alpha_k} \\
		\forall j \in \{1,\ldots,k\}: 2 \neq P_j
	\end{gather}
	V rozvoji $n^2$ není $2$, tak v rozvoji $n$ také není (liší se pouze mocninou).
\end{proof}
\begin{proof}[Důkaz 2]
	Chci: $\forall n \in \mathbb{N}: n$ sudé $\Rightarrow n^2$ sudé
	\begin{gather}
		n = 2k, k\in\mathbb{N} \\
		n^2 = 4k^2 = 2(2k^2)
	\end{gather}
\end{proof}
\begin{proof}[Důkaz 3]
	Předpokládejme: $n^2$ liché a $n$ sudé. Pak:
	\begin{gather}
		n^2 + n \text{ liché}  \\
		n(n + 1) \text{ liché a sudé zároveň (spor)}
	\end{gather}  
\end{proof}
O čem budou výroky? O definovaných pojmech:
\begin{itemize}
	\item množina: soubor prvků (př.: množina mužů, žen)
	\item $x\in A$ \quad $x$ je prvkem
	\item $x\notin A$ \quad $\lnot (x\in A)$
	\item $A\subset B$ \quad $A$ je podmnožinou $B$: $\forall x \in A: x \in B$
	\item $\emptyset$ \quad prázdná množina
	\item množinové operace:
	\begin{itemize}
		\item $A\cup B = \{x; (x\in A)\lor(x\in B)\}$
		\item $A\cap B = \{x; (x\in A)\land(x\in B)\}$
		\item $A - B = \{x; (x\in A)\lor(x\notin B)\}$
	\end{itemize}
	\item kvantifikátory:
	\begin{itemize}
		\item $\forall x$ \quad pro všechna $x$
		\item $\exists y$ \quad existuje $y$
		\item př.: $V(x,y)$ je vlastnost, že $y$ je matka $x$. $M$ je množina
			mužů, $Z$ je množina žen.
		\begin{itemize}
			\item $\forall x \in M~\exists y \in Z: V(x,y)$
			\item $\exists y \in Z: \forall x \in M: V(x,y)$
		\end{itemize}
	\end{itemize}
\end{itemize}

\section{Reálná čísla}

\begin{theoremAlph}[Reálná čísla]
	Existuje množina $\mathbb{R}$ s operacemi $\oplus$ a $\otimes$ a relací $<$ tak,
	že splňuje vlastnosti \textit{$A_1$} až \textit{$A_4$}.
\end{theoremAlph}

\begin{property}[Algebraická struktura]
	Vlastnosti:

	\begin{enumerate}[I]
		\item Komutativita: $\forall x, y \in \mathbb{R}: x + y = y + x;~x\cdot y = y\cdot x$
		\item Asociativita: $\forall x, y, z \in \mathbb{R}: x + (y + z) = (x + y) + z;~(x\cdot y)\cdot z = x\cdot (y\cdot z)$
		\item Nulový prvek $\oplus$, jednotka $\otimes$: $\exists~0 \in \mathbb{R}, \exists~1 \in \mathbb{R}:
		\forall x \in \mathbb{R}: 0 + x = x; 1 \cdot x = x$ 
		\item Inverzní prvek: $\forall x \in \mathbb{R}, \forall z \in \mathbb{R}~\exists ! y: x + y = z$ (právě jedno; ozn. $y = z - x$)\newline
		$\forall x, z \in \mathbb{R}, x\neq0 ~\exists ! y \in \mathbb{R}: x \cdot y = z$ (ozn. $y = z / x$)
		\item Distributivita: $\forall x, y, z \in \mathbb{R}: x (y + z) = xy + xz$
		\item Násobení nulou: $\forall x \in \mathbb{R}: 0\cdot x = 0$\newline
		$\forall x, y \in \mathbb{R}: x\cdot y = 0 \Rightarrow ((x = 0) \lor (y = 0))$
	\end{enumerate}
\end{property}
Další vlastnosti lze odvodit:
\begin{alignat}{1}
	-(-x) &= x \\
	-(x\cdot y) &= (-x)\cdot y
\end{alignat}
Další značení:
\begin{alignat}{1}
	x^n &= x\cdot x\cdot \text{...} \cdot x \text{($n$-krát)} \\
	-x &= 0 - x \\
	\forall x \neq 0: x^{-1} &= \frac{1}{x} \\
	\forall x \neq 0: x^{-n} &= \left( \frac{1}{x} \right)^n
\end{alignat}

I. - IV. říká $(\mathbb{R}, +)$ a $(\mathbb{R} - \{0\}, \cdot)$ jsou grupy.

I. - VI. říká $(\mathbb{R}, +, \cdot)$ je těleso.

Ověřte, že \textit{Vlastnost A1} platí pro $\mathbb{C}$ (komplexní čísla).

% \begin{figure}[ht!]
% 	\begin{center}
% 		\includegraphics[width=0.49\textwidth,keepaspectratio]{img/spintronika/SFET3.png}
% 		\includegraphics[width=0.49\textwidth,keepaspectratio]{img/spintronika/GMR.png}
% 		\caption{(a,b) Realizace \textit{field effect tranzistor} (spin-FET)[3]. (c)  na jevu GMR [4].}
% 	\end{center}
% \end{figure}\FloatBarrier
