\chapter{Kapitola 1: Úvod, základní pojmy}
Výrok - má pravdivostní hodnotu 0 nebo 1. Mějme A, B výroky:

\begin{figure}[ht!]
	\begin{center}
		\begin{tabular}{|C|C|C|C|C|C|C|}
			\hline
			& & A\land B & & & (A\Rightarrow B) \land (A\Leftarrow B) & \\
			A & B & A\&B & A\lor B & A\Rightarrow B & A\Leftrightarrow B & \lnot A \\
			\hline
			0 & 0 & 0 & 0 & 1 & 1 & 1 \\
			0 & 1 & 0 & 1 & 1 & 0 & 1 \\
			1 & 0 & 0 & 1 & 0 & 0 & 0 \\
			1 & 1 & 1 & 1 & 1 & 1 & 0 \\
			\hline
		\end{tabular}
		\caption{Tabulka pravdivostních hodnot}
	\end{center}
\end{figure}

Důkaz implikace $A\Rightarrow B$:
\begin{enumerate}
	\item přímý: ukážeme, že když $A = 1$, pak $B = 1$
	\item nepřímý: plyne z $\lnot B \Rightarrow \lnot A$
	\item sporem: předpokládáme, že $A = 1 \land B = 0$ a odvodíme spor (např.: $1=2$)
\end{enumerate}

\begin{lemma}
	(tvrzení) $\forall n\in \mathbb{N} : n^2~\text{liché} \Rightarrow n~\text{liché}$
\end{lemma}

\begin{proof}[Důkaz 1]
	Fixuj $n\in \mathbb{N}$. Prvočíselný rozklad:
	\begin{equation}
		n = p_1^{\alpha_1} \cdot \ldots \cdot p_k^{\alpha_k}
	\end{equation}
	\begin{equation}
		n^2 = p_1^{2\alpha_1} \cdot \ldots \cdot p_k^{2\alpha_k}
	\end{equation}
	\begin{equation}
		\forall j \in \{1,\ldots,k\}: 2 \neq P_j
	\end{equation}
	V rozvoji
\end{proof}

% \begin{figure}[ht!]
% 	\begin{center}
% 		\includegraphics[width=0.49\textwidth,keepaspectratio]{img/spintronika/SFET3.png}
% 		\includegraphics[width=0.49\textwidth,keepaspectratio]{img/spintronika/GMR.png}
% 		\caption{(a,b) Realizace \textit{field effect tranzistor} (spin-FET)[3]. (c)  na jevu GMR [4].}
% 	\end{center}
% \end{figure}\FloatBarrier
